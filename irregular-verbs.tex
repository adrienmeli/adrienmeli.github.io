% Options for packages loaded elsewhere
\PassOptionsToPackage{unicode}{hyperref}
\PassOptionsToPackage{hyphens}{url}
%
\documentclass[
]{article}
\usepackage{lmodern}
\usepackage{amssymb,amsmath}
\usepackage{ifxetex,ifluatex}
\ifnum 0\ifxetex 1\fi\ifluatex 1\fi=0 % if pdftex
  \usepackage[T1]{fontenc}
  \usepackage[utf8]{inputenc}
  \usepackage{textcomp} % provide euro and other symbols
\else % if luatex or xetex
  \usepackage{unicode-math}
  \defaultfontfeatures{Scale=MatchLowercase}
  \defaultfontfeatures[\rmfamily]{Ligatures=TeX,Scale=1}
\fi
% Use upquote if available, for straight quotes in verbatim environments
\IfFileExists{upquote.sty}{\usepackage{upquote}}{}
\IfFileExists{microtype.sty}{% use microtype if available
  \usepackage[]{microtype}
  \UseMicrotypeSet[protrusion]{basicmath} % disable protrusion for tt fonts
}{}
\makeatletter
\@ifundefined{KOMAClassName}{% if non-KOMA class
  \IfFileExists{parskip.sty}{%
    \usepackage{parskip}
  }{% else
    \setlength{\parindent}{0pt}
    \setlength{\parskip}{6pt plus 2pt minus 1pt}}
}{% if KOMA class
  \KOMAoptions{parskip=half}}
\makeatother
\usepackage{xcolor}
\IfFileExists{xurl.sty}{\usepackage{xurl}}{} % add URL line breaks if available
\IfFileExists{bookmark.sty}{\usepackage{bookmark}}{\usepackage{hyperref}}
\hypersetup{
  pdftitle={Les verbes irréguliers anglais},
  pdfauthor={Adrien Méli},
  hidelinks,
  pdfcreator={LaTeX via pandoc}}
\urlstyle{same} % disable monospaced font for URLs
\usepackage[margin=1in]{geometry}
\usepackage{longtable,booktabs}
% Correct order of tables after \paragraph or \subparagraph
\usepackage{etoolbox}
\makeatletter
\patchcmd\longtable{\par}{\if@noskipsec\mbox{}\fi\par}{}{}
\makeatother
% Allow footnotes in longtable head/foot
\IfFileExists{footnotehyper.sty}{\usepackage{footnotehyper}}{\usepackage{footnote}}
\makesavenoteenv{longtable}
\usepackage{graphicx}
\makeatletter
\def\maxwidth{\ifdim\Gin@nat@width>\linewidth\linewidth\else\Gin@nat@width\fi}
\def\maxheight{\ifdim\Gin@nat@height>\textheight\textheight\else\Gin@nat@height\fi}
\makeatother
% Scale images if necessary, so that they will not overflow the page
% margins by default, and it is still possible to overwrite the defaults
% using explicit options in \includegraphics[width, height, ...]{}
\setkeys{Gin}{width=\maxwidth,height=\maxheight,keepaspectratio}
% Set default figure placement to htbp
\makeatletter
\def\fps@figure{htbp}
\makeatother
\usepackage[normalem]{ulem}
% Avoid problems with \sout in headers with hyperref
\pdfstringdefDisableCommands{\renewcommand{\sout}{}}
\setlength{\emergencystretch}{3em} % prevent overfull lines
\providecommand{\tightlist}{%
  \setlength{\itemsep}{0pt}\setlength{\parskip}{0pt}}
\setcounter{secnumdepth}{5}
\usepackage{fontawesome}
\usepackage{amsmath}
\usepackage{tipa}
\usepackage{moodle}
\usepackage{hyperref}
\usepackage{booktabs}
\usepackage[utf8]{inputenc}
\usepackage[light,sfdefault]{roboto}
%\usepackage{fourier}
%\usepackage{montserrat}
%\usepackage[T1]{fontenc}
%\usepackage[french]{babel}

% -----------------------------------------------------------------
% Hyper Setup
% -----------------------------------------------------------------
\hypersetup{
    %bookmarks=true,         % show bookmarks bar?
    unicode=false,          % non-Latin characters in Acrobat�s bookmarks
    pdftoolbar=true,        % show Acrobat�s toolbar?
    pdfmenubar=true,        % show Acrobat�s menu?
    pdffitwindow=false,     % window fit to page when opened
    pdfstartview={FitH},    % fits the width of the page to the window
    %pdftitle={},    % title
    pdfauthor={Adrien Meli},     % author 
    pdfsubject={Phonological rules},   % subject of the document
    pdfcreator={Creator},   % creator of the document
    pdfproducer={Producer}, % producer of the document
    pdfkeywords={Second Language Acquisition, French, English, phonology}, % list of keywords
    pdfnewwindow=true,     % links in new window
    colorlinks=true,       % false: boxed links; true: colored links
    linkcolor=black,          % color of internal links
    citecolor=black,        % color of links to bibliography
    filecolor=magenta,      % color of file links
    urlcolor=blue,           % color of external links
    bookmarksopen=false,
    anchorcolor=black,
    bookmarksnumbered=true,
    pdfpagemode=UseOutlines,    %None/UseOutlines/UseThumbs/FullScreen
    linktocpage=true
}

% ********************Captions and Hyperreferencing / URL **********************

% Captions: This makes captions of figures use a boldfaced small font.

\usepackage[margin=10pt,font=small,labelfont=bf,labelsep=endash]{caption}

% -----------------------------------------------------------------
% TABLE OF CONTENTS
% -----------------------------------------------------------------
\usepackage[dotinlabels]{titletoc}
\titlecontents{section}[0em] % entries are pushed to the rtight
  {} % code to change the appearance
  {} % section number: increase distance to push to the left
  {\hspace*{3.3em}}
  {\titlerule*[1.9mm]{.}\contentspage}
% remove subsections from TOC
\setcounter{tocdepth}{1}

% multi-line curly brackets

\newenvironment{rightbracedtext}
 {$\kern-\nulldelimiterspace\left.\begin{tabular}{@{}l@{}}}
 {\end{tabular}\right\}$}

\newenvironment{leftbracedtext}{$\left\{\begin{tabular}{@{}l}}{\end{tabular}\right.$}

%----------------------------------------------------------------------------------------
%	MARGINS
%----------------------------------------------------------------------------------------

%\usepackage{geometry} % Required for adjusting page dimensions and margins
%
%\geometry{
%	paper=a4paper, % Change to letterpaper for US letter
%	top=3cm, % Top margin
%	bottom=3cm, % Bottom margin
%	left=2cm, % Left margin
%	right=3cm, % Right margin
%	headheight=14pt, % Header height
%	footskip=1.4cm, % Space from the bottom margin to the baseline of the footer
%	headsep=1.2cm, % Space from the top margin to the baseline of the header
%	%showframe, % Uncomment to show how the type block is set on the page
%}

\title{Les verbes irréguliers anglais}
\author{Adrien Méli}
\date{April 15, 2020}

\begin{document}
\maketitle

{
\setcounter{tocdepth}{1}
\tableofcontents
}
\hypertarget{principes}{%
\section{Principes}\label{principes}}

Les verbes irréguliers s'apprennent en trois colonnes :

\begin{itemize}
\item
  La \textbf{Base Verbale} (BV)
\item
  Le \textbf{Prétérit}
\item
  Le \textbf{Participe passé}
\end{itemize}

\hypertarget{la-base-verbale-bv}{%
\subsection{La Base Verbale (BV)}\label{la-base-verbale-bv}}

Précédée de \emph{to}, la Base Verbale constitue ce que l'on appelle l'infinitif.

À noter que le \textbf{présent simple}, utilisé pour parler des habitudes et vérités générales, ressemble à s'y méprendre à la BV.
La seule différence visible est à la troisième personne du singulier, où le verbe prend un \emph{\textless-s\textgreater{}}\footnote{Ceci est valable pour tous les verbes sauf \emph{BE}, \emph{HAVE}, et les 5 auxiliaires modaux, qui sont invariables.}.

\hypertarget{le-pruxe9tuxe9rit}{%
\subsection{Le prétérit}\label{le-pruxe9tuxe9rit}}

Des trois formes, le prétérit est la seule qui soit véritablement \textbf{conjuguée}.

On utilise le prétérit pour parler du passé.

Pour rappel, pour qu'une phrase soit correcte, il faut qu'elle ait un groupe nominal sujet qui gouverne un verbe conjugué.

Prenons par exemple le verbe \emph{forget} (\emph{forgot, forgotten}) :

\begin{enumerate}
\def\labelenumi{\arabic{enumi}.}
\tightlist
\item
  \sout{\textcolor{red}{*John forget*}}
\item
  \textbf{\textcolor{green}{*John forgot*}}
\item
  \sout{\textcolor{red}{*John forgotten*}}
\end{enumerate}

Des trois suggestions ci-dessus, seule la deuxième, \textbf{\textcolor{green}{*John forgot*}} (``John a oublié''), est correcte.

\hypertarget{le-participe-passuxe9}{%
\subsection{Le participe passé}\label{le-participe-passuxe9}}

Le participe passé s'emploie avec \emph{HAVE} ou \emph{BE}.

\begin{itemize}
\tightlist
\item
  \textbf{Avec \emph{HAVE} :} Le participe passé permet de renvoyer à une action qui est envisagée comme passée.
  Le fait qu'il ne soit pas conjugué permet de combiner ce renvoi au passé avec l'expression d'hypothèses ou de situations irréelles.

  \begin{itemize}
  \tightlist
  \item
    Exemple : \emph{John must have forgotten} \(\rightarrow\) ``John a dû oublier''.
  \end{itemize}
\item
  \textbf{Avec \emph{BE} :} le participe passé revêt alors un \textbf{\textcolor{red}{sens passif}}. Le sujet du verbe subit l'action.

  \begin{itemize}
  \tightlist
  \item
    Exemple : \emph{The book was written in 1932} \(\rightarrow\) ``Le livre a été écrit en 1932''.
  \end{itemize}
\end{itemize}

\begin{center}\rule{0.5\linewidth}{0.5pt}\end{center}

\textbf{\textcolor{red}{Rappel :}}

\begin{quote}
Les verbes réguliers fonctionnent selon les mêmes principes.

Ils sont appelés ``réguliers'' parce que leur prétérit et leur participe passé se construisent en ajoutant le suffixe \textless-(e)d\textgreater{} à leur BV :

\begin{itemize}
\item
  \emph{play} \(\rightarrow\) \emph{played, played}
\item
  \emph{decide} \(\rightarrow\) \emph{decided, decided}
\end{itemize}
\end{quote}

\begin{center}\rule{0.5\linewidth}{0.5pt}\end{center}

\hypertarget{tableau-des-verbes-irruxe9guliers}{%
\section{Tableau des verbes irréguliers}\label{tableau-des-verbes-irruxe9guliers}}

\begin{longtable}{>{\bfseries\raggedright\arraybackslash}p{3cm}ll>{\bfseries\raggedright\arraybackslash}p{3cm}}
\toprule
Base Verbale & Prétérit & Participe passé & Traduction\\
\midrule
\rowcolor{gray!6}  beat & beat & beaten & battre (coeur)\\

become & became & become & devenir\\

\rowcolor{gray!6}  begin & began & begun & commencer\\

bend & bent & bent & (se) courber\\

\rowcolor{gray!6}  bet & bet & bet & parier\\

bind & bound & bound & relier (livre)\\

\rowcolor{gray!6}  bite & bit & bitten & mordre\\

bleed & bled & bled & saigner\\

\rowcolor{gray!6}  blow & blew & blown & souffler\\

break & broke & broken & casser\\

\rowcolor{gray!6}  breed & bred & bred & élever\\

bring & brought & brought & apporter\\

\rowcolor{gray!6}  build & built & built & construire\\

burn & burnt & burnt & brûler\\

\rowcolor{gray!6}  buy & bought & bought & acheter\\

catch & caught & caught & attraper\\

\rowcolor{gray!6}  choose & chose & chosen & choisir\\

come & came & come & venir\\

\rowcolor{gray!6}  cost & cost & cost & coûter\\

cut & cut & cut & couper\\

\rowcolor{gray!6}  do & did & done & faire (auxiliaire)\\

dig & dug & dug & creuser\\

\rowcolor{gray!6}  draw & drew & drawn & dessiner\\

dream & dreamt & dreamt & rêver\\

\rowcolor{gray!6}  drink & drank & drunk & boire\\

drive & drove & driven & conduire\\

\rowcolor{gray!6}  eat & ate & eaten & manger\\

fall & fell & fallen & tomber\\

\rowcolor{gray!6}  feed & fed & fed & nourrir\\

feel & felt & felt & (res)sentir\\

\rowcolor{gray!6}  fight & fought & fought & se battre\\

find & found & found & trouver\\

\rowcolor{gray!6}  fly & flew & flown & voler (oiseau)\\

forget & forgot & forgotten & oublier\\

\rowcolor{gray!6}  forgive & forgave & forgiven & pardonner\\

freeze & froze & frozen & geler\\

\rowcolor{gray!6}  get & got & got & obtenir\\

give & gave & given & donner\\

\rowcolor{gray!6}  go & went & gone & aller\\

grow & grew & grown & grandir\\

\rowcolor{gray!6}  have & had & had & avoir\\

hear & heard & heard & entendre\\

\rowcolor{gray!6}  hide & hid & hidden & cacher\\

hit & hit & hit & frapper\\

\rowcolor{gray!6}  hold & held & held & tenir\\

hurt & hurt & hurt & faire mal\\

\rowcolor{gray!6}  keep & kept & kept & garder\\

know & knew & known & savoir\\

\rowcolor{gray!6}  lay & laid & laid & poser\\

lead & led & led & mener\\

\rowcolor{gray!6}  lean & leant & leant & pencher\\

leave & left & left & quitter\\

\rowcolor{gray!6}  lend & lent & lent & prêter\\

let & let & let & laisser\\

\rowcolor{gray!6}  lose & lost & lost & perdre\\

make & made & made & fabriquer\\

\rowcolor{gray!6}  mean & meant & meant & signifier\\

meet & met & met & rencontrer\\

\rowcolor{gray!6}  pay & paid & paid & payer\\

put & put & put & mettre\\

\rowcolor{gray!6}  quit & quit & quit & arrêter\\

read & read & read & lire\\

\rowcolor{gray!6}  ride & rode & ridden & aller en (véhicule)\\

ring & rang & rung & sonner\\

\rowcolor{gray!6}  rise & rose & risen & monter\\

run & ran & run & courir\\

\rowcolor{gray!6}  say & said & said & dire\\

see & saw & seen & voir\\

\rowcolor{gray!6}  sell & sold & sold & vendre\\

send & sent & sent & envoyer\\

\rowcolor{gray!6}  set & set & set & régler (machine)\\

shake & shook & shaken & secouer\\

\rowcolor{gray!6}  shine & shone & shone & briller\\

shoe & shod & shod & chausser (rare)\\

\rowcolor{gray!6}  shoot & shot & shot & tirer\\

show & showed & shown & montrer\\

\rowcolor{gray!6}  shrink & shrank & shrunk & rétrécir\\

shut & shut & shut & fermer\\

\rowcolor{gray!6}  sing & sang & sung & chanter\\

sink & sank & sunk & couler\\

\rowcolor{gray!6}  sit & sat & sat & s'asseoir\\

sleep & slept & slept & dormir\\

\rowcolor{gray!6}  speak & spoke & spoken & parler\\

spend & spent & spent & dépenser\\

\rowcolor{gray!6}  spill & spilt & spilt & renverser\\

spread & spread & spread & répandre\\

\rowcolor{gray!6}  speed & sped & sped & aller très/trop vite\\

stand & stood & stood & se tenir debout\\

\rowcolor{gray!6}  steal & stole & stolen & voler (crime)\\

stick & stuck & stuck & planter\\

\rowcolor{gray!6}  sting & stung & stung & piquer\\

stink & stank & stunk & puer\\

\rowcolor{gray!6}  swear & swore & sworn & jurer\\

sweep & swept & swept & balayer\\

\rowcolor{gray!6}  swim & swam & swum & nager\\

swing & swung & swung & osciller\\

\rowcolor{gray!6}  take & took & taken & prendre\\

teach & taught & taught & enseigner\\

\rowcolor{gray!6}  tear & tore & torn & déchirer\\

tell & told & told & raconter\\

\rowcolor{gray!6}  think & thought & thought & penser\\

throw & threw & thrown & jeter\\

\rowcolor{gray!6}  understand & understood & understood & comprendre\\

wake & woke & woken & réveiller\\

\rowcolor{gray!6}  wear & wore & worn & porter (vêtement)\\

win & won & won & gagner\\

\rowcolor{gray!6}  write & wrote & written & écrire\\
\bottomrule
\end{longtable}

\end{document}
